\documentclass[a4paper, titlepage]{article}

\usepackage{courier} % Required for the courier font
\usepackage{listings}
\lstset{showstringspaces=false, numbers=left, frame=single, breaklines=true, firstnumber=1}
\usepackage[utf8]{inputenc}
\usepackage{graphicx}
\usepackage[bookmarks]{hyperref}

\begin{document}

\title{BOSC2013 OO3 - Garbage Collection}
\author{Sigurt Bladt Dinesen \\sidi{@}itu.dk}
\maketitle
\tableofcontents{}
\pagebreak

\section{Re-handin}
Exercise 6 has been completed, and is described in section \ref{sec:Ex6}

\section{Exercise 1}
\subsection{Exercise 1 (i)}
\textit{Write 3-10 line descriptions of how the abstract machine executes ADD,
CSTI i, NIL, IFZERO, CONS, CAR , and SETCAR.}\\

$sp$ denotes the stack pointer, and $s[i]$ the element in the $i$'th slot of of the stack.
\begin{itemize}
	\item ADD - $s[sp-1]$ is set to the sum of $s[sp]$ and $s[sp-1]$, $sp$
		is decremented by 1.
	\item CSTI i - the stack machine pushes the integer i onto the stack,
		and increments $sp$. To indicate that i is an integer rather
		than a reference, i's binary representation is shifted one bit
		to the left, and the rightmost bit is set to 1. We refer to this
		as "tagging".
	\item NIL - the stack machine pushes the untagged integer (I.E. a
		reference) 0 onto the stack.
	\item IFZERO - the stack machine checks if $s[sp]$ is 0 or NIL, by
		first untagging it if it is an integer. It then sets the program
		counter to the "argument" of IFZERO if $s[sp]$ is zero or NIL,
		and increments it by one otherwise. The stack pointer is
		decremented by one regardless.
	\item CONS - the stack machine allocates three words on the heap, and
		sets the non-header words to $s[sp-1]$ and $s[sp]$ respectively.
	\item CAR - interprets $s[sp]$ as a word array, raising an error if it
		is zero, and otherwise replaces it with the first word in the
		array. $sp$ is decremented by one.
	\item SETCAR - the stack machine assumes $s[sp]$ is a word and that
		$s[sp-1]$ is a word array. Then sets the first non-header word
		of $s[sp-1]$ to $s[sp]$ and decrements $sp$.
\end{itemize}
\subsection{Exercise 1 (ii)}
\textit{Describe the C macros Length, Color, and Paint, and their effect on the
32-bit header blocks}\\

\begin{itemize}
	\item Length(hdr) - shifts hdr 2 bits to the right, in effect
		ignoring the color bits, and applies the binary and operation to
		the result and the binary value consisting of 22 ones. The
		result is the value of the bits 9 though 30 i.e the length bits.
	\item Color(hdr) - By applying binary and to hdr and 3 (11 in binary)
		the value of the last two bits is obtained - the color bits.
	\item Paint(hdr, color) - applying binary and to hdr and the result of
		not 3 returns hdr with both color bits set to zero. Applying
		binary or to that, and color, returns hdr with the color bits
		set to the input color.
\end{itemize}
\subsection{Exercise 1 (iii)}
\textit{When does the abstract machine (or mutator) interact with the garbage
collector?}\\

The stack machine directly calls the \texttt{allocate} method when it puts cons
cells on the heap. The \texttt{allocate} method may then call \texttt{collect},
which frees up heap space.

\subsection{Exercise 1 (iv)}
\textit{Under what circumstances is \emph{\texttt{collect}} called}\\

\texttt{collect} is called whenever the list machine attempts to allocate more
data than can be fit into a block on the free-list. I.e. when there is no more
memory available on the heap.

\section{Exercise 2}
\textit{Implement the "mark and sweep" algorithm for the listmachine}\\

\subsection{The "mark" phase}
My first implementation of the mark phase stepped through the stack, looking for
references to the heap. It did so under the assumption that if a stack element
is not an integer, it is a
heap reference.
For each heap reference on the stack, a recursive function \texttt{mark} was
called to color the referenced block white (mark it), and do the same to
any heap references that block may have contained.

\subsection{The "sweep" phase}
The sweep phase steps though every block on the heap, and puts white (marked)
blocks on the free-list. Black blocks (in-use blocks) are colored white, so that
the next sweep will consider them free unless they are reached in the next mark
phase.

An auxiliary function called \texttt{freeBlock} is used for putting blocks on
the free-list. It does so by treating the list as a linked list (because it is)
and then using the standard way of inserting elements into an ordered linked
list. This means the time complexity of \texttt{freeBlock} is $\Theta(n)$ in the number
of blocks allready on the free-list.\footnote{$\Theta$ as in Bachmann–Landau}
This makes the \texttt{sweepPhase} function $\Theta(n^2)$ in the number of free
blocks. Alternatively, the \texttt{sweepPhase} function could discard and
rebuild the free-list anew, allowing a complexity of $\Theta(n)$.

\section{Exercise 3 and 4}
\textit{Joining adjacent free blocks}\\

This is done by by \texttt{sweepPhase} letting the function \texttt{joinFree}
determine the how big a step to take through the heap. \texttt{joinFree}
increases the size bits of the header-block it takes as argument, so that it is
joined with any free blocks immediately following it. It then returns the size
of the new block to be the stepping value of \texttt{sweepPhase}.

\section{Exercise 5}
\label{sec:Ex5}
\textit{Implement the \texttt{markPhase} in a non-recursive fashion}\\

The non-recursive \texttt{markPhase} function iterates through the stack,
painting referred headers grey. It then invokes the \texttt{traverse} function
until there are no grey headers on the heap.

\texttt{traverse} iterates through the heap, painting grey headers black, and
painting any headers they refer grey.
It returns \texttt{TRUE} or \texttt{FALSE}, indicating whether there are still
any grey blocks on the heap.

\section{Exercise 6}
\label{sec:Ex6}
\textit{Implement "stop and copy" garbage collection}\\

The "stop and copy" allocate method, \texttt{allocateStopAndCopy} works simply
by putting the requested block in the first available slot in the free-list
(which is just the last, free, part of the from-heap). If there are not enough
free words, \texttt{collectStopAndCopy} is called.

\texttt{collectStopAndCopy} iterates through the heap, invoking
\texttt{copyBlock} on each heap reference it finds, replacing the stack content
with the result.

\texttt{copyBlock} takes a pointer to a block header, copies the entire block to
the free-list (which is now on the to-heap), and returns it new address on the to-heap.

\texttt{collectStopAndCopy} then invokes \texttt{copyFromTo}, which iterates
through every occupied block on the to-heap, looking for references to the
to-heap (live blocks that have not yet been copied). \texttt{copyFromTo} then
moves such refered blocks to the to-heap, marking the block on the from-heap
with a "forward" pointer to the to-heap, so that it can avoid copying the same
block more than once.
Finally, \texttt{copyFromTo} swaps the to- and from-heap.

\section{Testing}
To test the stop and copy implementation, simply make the default make target,
and run the executable listmachine on the exXX.out files.

Testing the mark and sweep implementation takes a little work (I know, I'm
sorry). To test mark and sweep, the following lines in listmachine.c need to be
commented in/out respectively: 638/639 and 303/304. Then the make/run process
is the same as for stop and copy.

\pagebreak
\section{Appendix/code}
listmachine.c
\lstinputlisting{listmachine.c}

\end{document}
