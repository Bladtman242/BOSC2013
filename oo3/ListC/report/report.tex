\documentclass[a4paper, titlepage]{article}

\usepackage{courier} % Required for the courier font
\usepackage{listings}
\lstset{showstringspaces=false, numbers=left, frame=single, breaklines=true, firstnumber=1}
\usepackage[utf8]{inputenc}
\usepackage{graphicx}
\usepackage[bookmarks]{hyperref}

\begin{document}

\title{BOSC2013 OO3 - Garbage Collection}
\author{Sigurt Bladt Dinesen \\sidi{@}itu.dk}
\maketitle
\tableofcontents{}
\pagebreak

\section{Exercise 1}
\subsection{Exercise 1 (i)}
\textit{Write 3-10 line descriptions of how the abstract machine executes ADD,
CSTI i, NIL, IFZERO, CONS, CAR , and SETCAR.}\\

$sp$ denotes the stack pointer, and $s[i]$ the element in the $i$'th slot of of the stack.
\begin{itemize}
	\item ADD - $s[sp-1]$ is set to the sum of $s[sp]$ and $s[sp-1]$, $sp$
		is decremented by 1.
	\item CSTI i - the stack machine pushes the integer i onto the stack,
		and increments $sp$. To indicate that i is an integer rather
		than a reference, i's binary representation is shifted one bit
		to the left, and the rightmost bit is set to 1. We refer to this
		as "tagging".
	\item NIL - the stack machine pushes the untagged integer (I.E. a
		reference) 0 onto the stack.
	\item IFZERO - the stack machine checks if $s[sp]$ is 0 or NIL, by
		first untagging it if it is an integer. It then sets the program
		counter to the "argument" of IFZERO if $s[sp]$ is zero or NIL,
		and increments it by one otherwise. The stack pointer is
		decremented by one regardless.
	\item CONS - the stack machine allocates three words on the heap, and
		sets the non-header words to $s[sp-1]$ and $s[sp]$ respectively.
	\item CAR - interprets $s[sp]$ as a word array, raising an error if it
		is zero, and otherwise replaces it with the first word in the
		array. $sp$ is decremented by one.
	\item SETCAR - the stack machine assumes $s[sp]$ is a word and that
		$s[sp-1]$ is a word array. Then sets the first non-header word
		of $s[sp-1]$ to $s[sp]$ and decrements $sp$.
\end{itemize}
\subsection{Exercise 1 (ii)}
\textit{Describe the C macros Length, Color, and Paint, and their effect on the
32-bit header blocks}\\

\begin{itemize}
	\item Length(hdr) - shifts hdr 2 bits to the right, in effect
		ignoring the color bits, and applies the binary and operation to
		the result and the binary value consisting of 22 ones. The
		result is the value of the bits 9 though 30 i.e the length bits.
	\item Color(hdr) - By applying binary and to hdr and 3 (11 in binary)
		the value of the last two bits is obtained - the color bits.
	\item Paint(hdr, color) - applying binary and to hdr and the result of
		not 3 returns hdr with both color bits set to zero. Applying
		binary or to that, and color, returns hdr with the color bits
		set to the input color.
\end{itemize}
\subsection{Exercise 1 (iii)}
\textit{When does the abstract machine (or mutator) interact with the garbage
collector?}\\

The stack machine directly calls the \texttt{allocate} method when it puts cons
cells on the heap. The \texttt{allocate} method may then call \texttt{collect},
which frees up heap space.

\subsection{Exercise 1 (iv)}
\textit{Under what circumstances is \emph{\texttt{collect}} called}\\

\texttt{allocate} calls \texttt{collect} if an attempt is made to allocate more
heap space than is in the free list.
\section{Exercise 2}
\textit{Implement the "mark and sweep" algorithm for the listmachine}\\

\subsection{The "mark" phase}
\subsection{The "sweep" phase}
\end{document}
