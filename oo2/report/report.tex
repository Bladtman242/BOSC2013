\documentclass[a4paper, titlepage]{article}

\usepackage{courier} % Required for the courier font
\usepackage{listings}
\lstset{showstringspaces=false, numbers=left, frame=single, breaklines=true, firstnumber=0}
\usepackage[utf8]{inputenc}

\begin{document}

\title{BOSC2013 OO2 - Asynchronous lists and stuff}
\author{Sigurt Bladt Dinesen \\sidi{@}itu.dk}
\maketitle
\section*{Opgaver og løsninger}
\subsection*{Parallelism}
\textit{Assume that the list implemetation is used in an asynchronous
environment. What issues might occur?}\\

Since \emph{list\_remove} and \emph{list\_add} work on the same list (albeit on
different ends of it), raise conditions would occur if different threads of
execution were to run these methods simultaneously. In fact, different threads
running just \textit{one} of these methods would be problematic. The trivial
example is that of reading from, and writing to, a variable, e.g. to increment
its value by a constant.

The execution of such example is illustrated in figure
\ref{tab:threadaccess}, where $T_2$ sets the shared entity to the result
applying a function to it, regardless of changes made by $T_1$. After
execution, $S$ is the result of applying $T_2.f$ to $\alpha$, when it should
have been the result of applying $T_2.f$ to $\alpha'$. The thing to notice here
is that $$T_2.f(\alpha) \neq T_2.f(\alpha') = T_2.f(T_1.f(\alpha))$$

\begin{table}[hbtp]
	\centering
	\begin{tabular}{|l|l|r|}
		\hline
		$T_1\ action$&$T_2\ action$&$S\ value$\\
		\hline
		&&$\alpha$\\
		\hline
		$T_1.a = S$&&$\alpha$\\
		\hline
		&$T_2.a = S$&$\alpha$\\
		\hline
		$S = T_1.f(S) = \alpha'$&&$T_1.f(\alpha)$\\
		\hline
		&$S = T_2.f(S)$&$T_2.f(\alpha)$\\
		\hline
	\end{tabular}
	\caption{
		Execution of two threads on a shared entity. The figure shows
		the actions of two threads, $T_1$ and $T_2$, operating on a
		shared value $S$.
		\label{tab:threadaccess}
	}
\end{table}


Similar problems would
occur for the next variables of the list's nodes.

\pagebreak
\lstinputlisting{prodcons.c}[name=prod]
\lstinputlisting{sumsqrt.c}
\lstinputlisting{list/main.c}
\lstinputlisting{list/list.c}
\lstinputlisting{makefile}
\lstinputlisting{list/Makefile}
\end{document}
