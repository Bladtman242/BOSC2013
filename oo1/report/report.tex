\documentclass[a4paper, titlepage]{article}

\usepackage{courier} % Required for the courier font
\usepackage{listings}
\lstset{showstringspaces=false}
\usepackage[utf8]{inputenc}

\begin{document}

\title{BOSC2013 OO1 - the BOSC shell}
\author{Sigurt Bladt Dinesen \\sidi{@}itu.dk}
\maketitle
\section*{Opgaver og løsninger}
\subsection*{Værtsnavn}
\textit{Kommando-prompten skal vise navnet på a den host den kører på}\\

Det sørger funktionen \emph{gethostname} for, som vist i appendix
\ref{gethostname}. Funktionen bruger biblioteksfunktionen \emph{fopen} til at
skabe en filstrøm fra \emph{/proc/sys/\linebreak[0]kernel/hostname}, der så bruges i et kald
til fgets, sammen med et array af arbitrær længde, og dettes længde. Dette
array, givet som parameter til funktionen, bliver da fyldt med den første (og
eneste) linje i filen \emph{hostname}.

\subsection*{Enkeltstående kommandoer}
\textit{ En bruger skal kunne indtaste almindelige enkeltstå aende kommandoer, så a som
	ls, cat og wc. Hvis kommandoen ikke findes skal der udskrives enCommand
	not found meddelelse.}\\

\emph{executecmd} (appendix \ref{forkexec}) bruger biblioteksfunktionen
\emph{fork} til at starte en ny process der, via biblioteksfunktionen
\emph{execvp}, erstater sig selv med programmet der beskrives af testkstrengen
indtastet af brugeren.
Magien sker i linje 23 og 29.

\subsection*{Baggrundsprocessor}
\textit{Kommandoer skal kunne eksekvere som baggrundsprocesser (ved brug af \&)
	sådan at mange programmer kan køres på a samme tid.}\\

I appendix \ref{forkexec}, linje 40-42, sørger programmet for kun at vente på
den forkede process, hvis bg ikke er sat på Shellcmd'en.

\subsection*{Piping}
\textit{Det skal være muligt at anvende pipes.}\\

Kommandoer behandles rekursivt.
Hvert kald opretter et rør, hvis ud-ende overskriver kommandoens stdin.
Ind-enden sendes med til den foregående kommando;
det næste lag af rekursion, der overskriver sin stdout med den.
Det er en simpel løsning, der har den ulempe at rekursionsstakken overflyder
hvis rør-kæden bliver for lang (25 kommandoer på mit system).
Alternativt kunne man køre en løkke (eller to) over komandolisten.

\subsection*{Redirection}
\textit{ Der skal være indbygget funktionalitet som gør de muligt at lave redirection
af stdin og stdout til filer.}\\

To af \emph{executecmd}'s parametre, std\_in og std\_out, sættes af den kaldende
funktion (\emph{executeshellcmd}, ej beskrevet i denne rapport) til de
relevante file descriptors. \emph{executecmd} overskriver den sidste kommandos
stdout med std\_out, og kopierer manuelt std\_in til den første komandos stdin.
Stdin delen er noget uelegant, og kan alternativt gøres ved et look-ahead når
der forkes. std\_in kunne så overskrive den første kommandos stdin ligesom det
det er tilfældet for resten af kommandoerne.

\subsection*{SIGINT}
\textit{ Tryk på a Ctrl-C skal afslutte det program, der kører i bosh shellen, men ikke
	shell'en selv.}\\

Dette gøres ved at ignorere SIGINT i bosh processen (implementeret ved et kald
til \emph{signal}i \emph{main}), og registrere en funktion, \emph{siginttrap},
(appendix \ref{siginttrap}) som handler i den forkede process. Det virker dog
ikke helt efter hensigten, da siginttrap aldrig printer til konsollen.

\newpage
\appendix
\section{Retrieving the system hostname}
\label{gethostname}
\begin{lstlisting}[language=C]
1 char* gethostname(char *hostname){
2   FILE* f = fopen("/proc/sys/kernel/hostname", "r");
3   char* stat = fgets(hostname, HOSTNAMEMAX, f);
4   fclose(f);
5   hostname[strlen(hostname)-1]=0;
6   return stat;
7 }
\end{lstlisting}

\pagebreak
\section{Forking and Executing a Command}
\label{forkexec}
\begin{lstlisting}[language=C, numbers=left, frame=single, breaklines=true]
int executecmd (Cmd* cmd, int std_in, int std_out, int bg){
  if(cmd == NULL){
    if(std_in != 0){
      puts("copying");
      void* buf[1024];
      int size=0;
      do {
        size = read(std_in, buf, 1024);
        write(std_out, buf, size);
        printf("did %d bytes, errno is: %d\n", size, errno);
      } while (size > 0);
      close (std_in);
    }
    close (std_out);
    return 0;
  }

  int pfds[2];
  pipe(pfds);

  executecmd(cmd->next, std_in, pfds[1], 0);

  pid_t child = fork();
  if(child==0){
    signal(SIGINT, siginttrap);
    dup2(pfds[0], 0);
    dup2(std_out, 1);
    close(pfds[1]);
    if(execvp(cmd->cmd[0], cmd->cmd)){
      printf("Error in: %s\n", cmd->cmd[0]);
      if(errno == 2){
        printf("%s not found\n", cmd->cmd[0]);
      }else {
        printf("errno set to: %d", errno);
      }
      exit (errno);
    }
  }

  if(std_out == 1 && !bg){
    waitpid(child, NULL, NULL);
  }
  else if (std_out != 1){
     close(std_out);
  }
  return 0;
}
\end{lstlisting}

\pagebreak
\section{A signal handler}
\label{siginttrap}
\begin{lstlisting}[language=C]
1 void siginttrap(int signal){
2   printf("\nI gots sigint %d\n", signal);
3   exit (0);
4 }
\end{lstlisting}

\pagebreak
\section{Code}
\label{forkexec}
\subsection{bosh.c}
\lstinputlisting{../bosh.c}
\pagebreak
\subsection{Makefile}
\lstinputlisting{../Makefile}

\end{document}
