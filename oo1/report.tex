\documentclass[a4paper, titlepage, draft]{article}

\usepackage{courier} % Required for the courier font
\usepackage[utf8]{inputenc}
\usepackage{verbatim}

\begin{document}

\title{BOSC2013 OO1 - the BOSC shell}
\author{Sigurt Bladt Dinesen \\sidi{@}itu.dk}
\maketitle
\section*{Forord}
\section*{Opgaver og løsninger}
\subsection*{Værtsnavn}
\textit{Kommando-prompten skal vise navnet på a den host den kører på}\\

Det sørger funktionen \emph{gethostname} for, som vist i appendix
\ref{gethostname}. Funktionen bruger biblioteksfunktionen \emph{fopen} til at
skabe en filstrøm fra\\\emph{/proc/sys/kernel/hostname}, der så bruges i et kald
til fgets, sammen med et array af arbitrær længde, og dettes længde. Dette
array, givet som parameter til funktionen, bliver da fyldt med den første (og
eneste) linje i filen \emph{hostname}.

\subsection*{Enkeltstående kommandoer}
\textit{ En bruger skal kunne indtaste almindelige enkeltstå aende kommandoer, så a som
	ls, cat og wc. Hvis kommandoen ikke findes skal der udskrives enCommand
	not found meddelelse.}\\

executecmd starts from the last cmd in the cmdline, and moves backwards,
hence, the previous command in the cmdline is the next cmd to be processed
by executecmd, and vice versa.
Comments in the function refers to the cmdline when using the terms "next"
and previous.
if there are no more commands,
close the pipe and return
make pipe, to bind the previous cmd's stdout to this one's stdin
setup the previous cmds recursively
bg is set to 0, as only the last cmd in the cmdline (first in terms of this function)
should be affected by backgrounding
fork and exec current cmd
copy the pipe's read part to cmd stdin,
next cmd's
then close the write part of pipe, we don't need it
there must be a better way of interpreting errnos?
if this is the last cmd (indicated by stdout being 1, this should be done
differently) and bg isn't set, wait for the this cmd
only close the stdout fd if it isn't the stdout of the program,
otherwise the REPL (yes, I called it REPL) would die
\newpage
\appendix
\section{Retrieving the system hostname}
\label{gethostname}
\begin{verbatim}
1 char* gethostname(char *hostname){
2   FILE* f = fopen("/proc/sys/kernel/hostname", "r");
3   char* stat = fgets(hostname, HOSTNAMEMAX, f);
4   fclose(f);
5   hostname[strlen(hostname)-1]=0;
6   return stat;
7 }
\end{verbatim}

\end{document}
